\documentclass[13pt]{beamer}

% Presento style file
\usepackage{config/presento}

% Remove "figure" label in figures
\setbeamertemplate{caption}{\raggedright\insertcaption\par}

% custom command and packages
% custom packages
\usepackage{textpos}
\setlength{\TPHorizModule}{1cm}
\setlength{\TPVertModule}{1cm}

\newcommand\crule[1][black]{\textcolor{#1}{\rule{2cm}{2cm}}}



% Information
\title{Branch \& Bound}
\author{Alberto Parravicini}
\institute{Université libre de Bruxelles}
\date{June 1, 2017}

\setbeamertemplate{bibliography item}[text] % No icons in bibliography

\begin{document}

% Title page
\begin{frame}[plain]
\maketitle
\end{frame}


% DEFINITIONS %%%%%%%%%%%%

\begin{frame}{A brief introduction...}  
    \begin{fullpageitemize}	
        \item<1->Consider a generic optimization problem:
        \item<2->$$\mathbf{max\{c(x): x \in X\}}$$
        \item<3->Often too hard to solve directly.
        \item<4->\textbf{Branch \& Bound} finds the optimal solutions thanks to:
        \begin{baseitemize}
            \item<5->[\rtarrow]\textbf{Branch}: find the optimal solution by exploring smaller sub-problems. 
            \item<6->[\rtarrow]\textbf{Bound}: use bounds on the objective function to implicitly enumerate all possible solutions.
        \end{baseitemize}	
    \end{fullpageitemize}
\end{frame}


\begin{frame}{Branching}
    \begin{fullpageitemize}
    \item<1->\textbf{Observation:} if we partition the feasible region $X$ into $\mathbf{X = X_1 \cup \ldots \cup X_k}$ and let $\mathbf{z_i = max\{c(x): x \in X_i\}}$, then:
    \item<2->[\rtarrow]$\mathbf{z = max_{1 \leq i \leq z_i}}$
    \item<3->Each subset $X_i$ can be split even further, giving an enumeration tree.
    \item<4->We can't explore all possible branches though, they are too many!   
    \end{fullpageitemize}
\end{frame}

\begin{frame}{Bounding}
    \begin{fullpageitemize}
        \item<1->We can use \textit{upper} and {lower} bounds to cut portions of the feasible region $\mathbf{X}$.
        \item<2->\textbf{Observation:} given $\mathbf{X = X_1 \cup \ldots \cup X_k}$, if $\overline{z}_k$ is an \textit{upper} bound for $X_k$ then $max_k{\overline{z}_k\}}$ is an \textit{upper} bound for $X$. Also, if $\underline{z}_k$ is a \textit{lower} bound for $X_k$ then $max_k{\{\underline{z}_k\}}$ is a \textit{lower} bound for $X$.
        \item<3->\textit{Lower} bounds are feasible solutions, while \textit{upper} bounds can be found by solving a relaxation.\\
        In \textbf{ILP}, the linear relaxation will provide an upper bound $\mathbf{x^*_{LP}}$.
    \end{fullpageitemize}
\end{frame}    

\begin{frame}{Pruning criteria}
    \begin{fullpageitemize}
        \item<1->When can we prune a node of the tree?
        \begin{baseitemize}
            \item<2->[\rtarrow]\textbf{Optimality}: if $\overline{z}_i = \underline{z}_i$ then we have an optimal solution of $X_i$.
            \item<3->[\rtarrow]\textbf{Infeasibility}: if $\mathbf{X_i = \emptyset}$.
            \item<4->[\rtarrow]\textbf{Bounding}: if the \textbf{upper} bound $\overline{z}_i$ is $\mathbf{\leq}$ than some other \textbf{lower} bound $\underline{z}_j$, or of the best feasible solution found so far.
        \end{baseitemize}	
    \end{fullpageitemize}
\end{frame}    

%\begin{frame}{In practice...}
%    \begin{fullpageitemize}
%        \item<1->[\rtarrow]How are the \textbf{upper} bounds obtained?
%        \begin{baseitemize}
%            \item<2->In \textbf{ILP}, by solving the linear relaxation of the problem. In other cases, with a feasible dual solution or with a combinatorial relaxation (e.g. \textit{1-tree} for \textbf{TSP}).
%        \end{baseitemize}	
%        \item<3->[\rtarrow]How are the \textbf{lower} bounds obtained?
%        \begin{baseitemize}
%            \item<2->Either heuristically, or by finding a feasible primal solution.
%        \end{baseitemize}	
%    \end{fullpageitemize}
%\end{frame}   

\begin{frame}{In practice...}
    \begin{fullpageitemize}
        \item<1->[\rtarrow]\textbf{How are branches defined?}
        \begin{baseitemize}
            \item<2->In \textbf{ILP}, pick a fractional variable $\mathbf{x_h}$ in $\mathbf{x^*_{LP}}$, and define $2$ sub-problems:
            \item<3->$z^1_{ILP} = max\{cx: x \in X, \mathbf{x_h \leq \lfloor a_h \rfloor}\}$
            \item<3->$z^2_{ILP} = max\{cx: x \in X, \mathbf{x_h \geq \lceil a_h \rceil}\}$
        \end{baseitemize}	
        \item<4->[\rtarrow]\textbf{Which variable is chosen?}
        \begin{baseitemize}
            \item<5->$\bullet$ The most fractional.
            \item<6->$\bullet$ \textbf{Strong Branching}: try all of them, keep the one with highest upper bound.
            \item<7->$\bullet$ Other (smallest sub-tour, etc...)
        \end{baseitemize}	
        \item<8->[\rtarrow]\textbf{Which branch is chosen?}
        \begin{baseitemize}
            \item<9->$\bullet$ \textbf{Depth First}, quickly find a feasible solution; easy to re-optimize.
            \item<10->$\bullet$ \textbf{Best Bound}, find the upper bounds of the children and choose the best (slow!).
        \end{baseitemize}
    \end{fullpageitemize}
\end{frame}   


\begin{frame}{Further optimizations}
    \begin{fullpageitemize}
         \item<1->[\rtarrow]\textbf{Generalized upper bounds}
         \begin{baseitemize}
             \item<2->Used in constraints like $\sum_{i=1}^n{x_i}=1$; fixing one $x_i = 1$ creates unbalanced tree; instead, use:
             \item<3->$X_1 = \mathbf{x: x_i = 0,  i=0,\ldots, r}$
             \item<3->$X_2 = \mathbf{x: x_i = 0,  i=r+1,\ldots, n}$
             \item<3->$\mathbf{r = min\{t: \sum_{i=1}^t{x_i} >= \frac{1}{2}\}}$
         \end{baseitemize}	
         \item<4->[\rtarrow]\textbf{Constraints preprocessing}
         \item<5->[\rtarrow]\textbf{Branch \& Cut}
         \begin{baseitemize}
             \item<6->$\bullet$ Combine \textbf{Gomory cuts} with \textbf{Branch \& Bound}.
             \item<7->$\bullet$ At each node, add \textbf{cuts} to the relaxation of the sub-problem.
             \item<8->$\bullet$ Branch only when adding more cuts stops being effective.
             \item<9->$\bullet$ Computational tradeoff, as cuts are valid only for a node and its children. Very effective in practice, however.
         \end{baseitemize}
     \end{fullpageitemize}
 \end{frame}      
\framecard[colorgreen]{{\color{white}\hugetext{THANK YOU!}}}



\end{document}